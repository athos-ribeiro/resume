%%%%%%%%%%%%%%%%%%%%%%%%%%%%%%%%%%%%%%%%%
% Friggeri Resume/CV
% XeLaTeX Template
% Version 1.2 (3/5/15)
%
% This template has been downloaded from:
% http://www.LaTeXTemplates.com
%
% Original author:
% Adrien Friggeri (adrien@friggeri.net)
% https://github.com/afriggeri/CV
%
% License:
% CC BY-NC-SA 3.0 (http://creativecommons.org/licenses/by-nc-sa/3.0/)
%
% Important notes:
% This template needs to be compiled with XeLaTeX and the bibliography, if used,
% needs to be compiled with biber rather than bibtex.
%
%%%%%%%%%%%%%%%%%%%%%%%%%%%%%%%%%%%%%%%%%

\documentclass[]{friggeri-cv} % Add 'print' as an option into the square bracket to remove colors from this template for printing

\addbibresource{bibliography.bib} % Specify the bibliography file to include publications

\begin{document}

\header{Athos}{\ Ribeiro}{Software Engineer} % Your name and current job title/field

%----------------------------------------------------------------------------------------
%	SIDEBAR SECTION
%----------------------------------------------------------------------------------------

\begin{aside} % In the aside, each new line forces a line break
\section{contact}
Avenida Diógenes Ribeiro de Lima, 2001
São Paulo, SP 05458-001
Brazil
~
+55(11) 98437-7463
~
\href{mailto:athoscribeiro@gmail.com}{athoscribeiro AT gmail}
\href{http://athoscr.me}{athoscr.me}
\href{https://www.ime.usp.br/~athoscr}{ime.usp.br/$\textasciitilde$athoscr}
\href{https://github.com/athos-ribeiro}{github: athos-ribeiro}
\section{languages}
Portuguese - native
English - fluent
Spanish - basic
\section{programming}
Python, Ruby, Perl, Shell script, Go, C/C++, HTML5, SQL
\section{software}
GNU/Linux, Ansible, Chef, Docker, Vagrant, RPM, git, \LaTeX
\end{aside}

%----------------------------------------------------------------------------------------
%	EDUCATION SECTION
%----------------------------------------------------------------------------------------

\section{education}

\begin{entrylist}

%------------------------------------------------

\entry
{2016--current}
{Master's degree -- {\normalfont Computer Science}}
{University of São Paulo, Brazil}
  {\emph{Ranking Warnings Based on Continuous Source Code Static Analysis Reports} \\ This ongoing thesis aims to train a machine learning model to identify possible false positive warnings in static analysis reports from multiple tools and provide a ranked warning list to users.}

%------------------------------------------------

\entry
{2011--2015}
{Bachelor -- {\normalfont Software Engineering}}
{University of Brasília, Brazil}

%------------------------------------------------

\end{entrylist}

%----------------------------------------------------------------------------------------
%	WORK EXPERIENCE SECTION
%----------------------------------------------------------------------------------------

\section{experience}

\subsection{Work}

\begin{entrylist}

%------------------------------------------------

\entry
{2015--2016}
{LAPPIS}
{University of Brasília, Brazil}
{\emph{Software Developer} \\
  Worked as a software developer on Portal do Software Público
  Brasileiro - a platform for hosting and development of Brazilian government
  FLOSS. \\
Detailed achievements:
\begin{itemize}
  \item Worked with the DevOps team on a Continuous Delivery strategy
  \item Performed deploys in production environment
  \item Helped Python and Ruby teams debugging tool integration issues
\end{itemize}
  }

%------------------------------------------------

\entry
{2014--2015}
{NIST - National Institute of Standards and Technology}
{Gaithersburg, MD - USA}
{\emph{Guest Researcher} \\
  Worked as a Guest Researcher for the Software Assurance Metrics and Tool
  Evaluation (SAMATE) team in the Software and Systems Division. \\ Detailed
  achievements:
\begin{itemize}
\item Learned about the CWE and CVE projects
\item Investigated software flaws and static analyzer warnings
\end{itemize}
  }

%------------------------------------------------

\entry
{2013--2014}
{2Reach}
{Brasília, Brazil}
{\emph{Software Developer} \\
  Worked as a Ruby on Rails developer on Sizify, a function points analysis
  platform. \\
Detailed achievements:
\begin{itemize}
  \item Implemented a first version of the platform automated test suite
\end{itemize}}

%------------------------------------------------
\entry
{2013}
{LEI - Laboratório de Engenharia e Inovação}
{University of Brasília, Brazil}
{\emph{Network Administrator} \\
  Worked on the configuration and administrtion of the laboratory network. \\
}

%------------------------------------------------

\entry
{2013}
{CQTS - Centro de Qualidade e Testes de Software}
{University of Brasília, Brazil}
{\emph{Software Developer and System Administrator} \\
  Worked on the configuration and customization of OTRS for Brazilian
  government Ministries of Culture and Communications. \\
Detailed achievements:
\begin{itemize}
  \item Deployed our solution in both staging and production environments
\end{itemize}}

%------------------------------------------------

% This was needed to avoid weird page break behaviors :/
% this may need to be moved as new entries are added to this resume
\end{entrylist}
\begin{entrylist}

\entry
{2011}
{Concreta Consultoria e Serviços}
{University of Brasília, Brazil}
{\emph{Marketing Director} \\
Marketing Director of Concreta Consultoria e Serviços, Civil Engineering Junior
  Enterprise at Universidade de Brasília. \\
}

%------------------------------------------------

\end{entrylist}

\subsection{Free and Open Source Projects}

\begin{entrylist}

\entry
{2015--Now}
{Fedora Project}
{Remote}
{\emph{Fedora Project Contributor} \\
Detailed achievements:
  \begin{itemize}
    \item Packaging RPM software to be included in the distribution, which often involves sending patches to upstream projects
    \item Member of the Static Analysis Special Interest Group
    \item Member of the Ruby Special Interest Group
    \item Member of the x86 Special Interest Group
    \item Infrastructure Apprentice
    \item English to Brazilian Portuguese Translator
    \item Help organizing Fedora related events and meetings in Brazil
  \end{itemize}}

%------------------------------------------------

\entry
{2017}
{Google Summer of Code}
{Remote}
{\emph{Mentor for the Fedora Project} \\
  The project developed during the program was
  \href{pagure.io/kiskadee}{kiskadee} - a continuous source code static
  analysis tool. \\
 }

%------------------------------------------------

\end{entrylist}

%----------------------------------------------------------------------------------------
%	COMMUNICATION SKILLS SECTION
%----------------------------------------------------------------------------------------

\section{presentations}

\begin{entrylist}

%------------------------------------------------

\entry
{2017}
{Oral Presentation @ II Seminário "Software e Cultura no Brasil"}
{São Bernardo do Campo, SP, Brazil}
{Introduced the Fedora Project community and described the distribution release process.}

%------------------------------------------------

\entry
{2016}
{Oral Presentation @ FGSL - Fórum Goiano de Software Livre}
{Goiânia, GO, Brazil}
{Talked about the Fedora Project package review process and performed a live official package review during the presentation}

\entry
{2016}
{Oral Presentation @ FUDCon - Fedora Users and Developers Conference}
{Puno, Peru}
{During FUDCon Puno I presented a few talks, involving RPM packaging and software licenses.}

%------------------------------------------------

\entry
{2015}
{Oral Presentation @ FISL - Fórum Internacional de Software Livre}
{Porto Alegre, RS, Brazil}
{Presented a few lightining talks about our Continuous Delivery strategies in
  Portal do Software Público Brasileiro.}

%------------------------------------------------

\end{entrylist}

%----------------------------------------------------------------------------------------
%	INTERESTS SECTION
%----------------------------------------------------------------------------------------

\section{interests}

\textbf{professional:} Free Software, Static Analysis, Computer Security, DevOps, Software Design, Programming, Release Engineering, Infrastructure\\
\textbf{personal:} Soccer, Homebrewing, Brazilian Jiu-Jitsu, DIY

%----------------------------------------------------------------------------------------
%	PUBLICATIONS SECTION
%----------------------------------------------------------------------------------------

\section{publications}

\printbibsection{article}{article in peer-reviewed journal} % Print all articles from the bibliography

\printbibsection{book}{books} % Print all books from the bibliography

\begin{refsection} % This is a custom heading for those references marked as "inproceedings" but not containing "keyword=brazil"
\nocite{*}
\printbibliography[sorting=chronological, type=inproceedings, title={international peer-reviewed conferences/proceedings}, notkeyword={brazil}, heading=bibheading]
\end{refsection}

\begin{refsection} % This is a custom heading for those references marked as "inproceedings" and containing "keyword=brazil"
\nocite{*}
\printbibliography[sorting=chronological, type=inproceedings, title={local peer-reviewed conferences/proceedings}, keyword={brazil}, heading=bibheading]
\end{refsection}

\printbibsection{misc}{other publications} % Print all miscellaneous entries from the bibliography

\printbibsection{report}{research reports} % Print all research reports from the bibliography

%----------------------------------------------------------------------------------------

\end{document}
